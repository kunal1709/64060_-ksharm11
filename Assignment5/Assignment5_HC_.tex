% Options for packages loaded elsewhere
\PassOptionsToPackage{unicode}{hyperref}
\PassOptionsToPackage{hyphens}{url}
%
\documentclass[
]{article}
\usepackage{lmodern}
\usepackage{amssymb,amsmath}
\usepackage{ifxetex,ifluatex}
\ifnum 0\ifxetex 1\fi\ifluatex 1\fi=0 % if pdftex
  \usepackage[T1]{fontenc}
  \usepackage[utf8]{inputenc}
  \usepackage{textcomp} % provide euro and other symbols
\else % if luatex or xetex
  \usepackage{unicode-math}
  \defaultfontfeatures{Scale=MatchLowercase}
  \defaultfontfeatures[\rmfamily]{Ligatures=TeX,Scale=1}
\fi
% Use upquote if available, for straight quotes in verbatim environments
\IfFileExists{upquote.sty}{\usepackage{upquote}}{}
\IfFileExists{microtype.sty}{% use microtype if available
  \usepackage[]{microtype}
  \UseMicrotypeSet[protrusion]{basicmath} % disable protrusion for tt fonts
}{}
\makeatletter
\@ifundefined{KOMAClassName}{% if non-KOMA class
  \IfFileExists{parskip.sty}{%
    \usepackage{parskip}
  }{% else
    \setlength{\parindent}{0pt}
    \setlength{\parskip}{6pt plus 2pt minus 1pt}}
}{% if KOMA class
  \KOMAoptions{parskip=half}}
\makeatother
\usepackage{xcolor}
\IfFileExists{xurl.sty}{\usepackage{xurl}}{} % add URL line breaks if available
\IfFileExists{bookmark.sty}{\usepackage{bookmark}}{\usepackage{hyperref}}
\hypersetup{
  pdftitle={Assignment5\_HC},
  pdfauthor={Kunal Sharma},
  hidelinks,
  pdfcreator={LaTeX via pandoc}}
\urlstyle{same} % disable monospaced font for URLs
\usepackage[margin=1in]{geometry}
\usepackage{color}
\usepackage{fancyvrb}
\newcommand{\VerbBar}{|}
\newcommand{\VERB}{\Verb[commandchars=\\\{\}]}
\DefineVerbatimEnvironment{Highlighting}{Verbatim}{commandchars=\\\{\}}
% Add ',fontsize=\small' for more characters per line
\usepackage{framed}
\definecolor{shadecolor}{RGB}{248,248,248}
\newenvironment{Shaded}{\begin{snugshade}}{\end{snugshade}}
\newcommand{\AlertTok}[1]{\textcolor[rgb]{0.94,0.16,0.16}{#1}}
\newcommand{\AnnotationTok}[1]{\textcolor[rgb]{0.56,0.35,0.01}{\textbf{\textit{#1}}}}
\newcommand{\AttributeTok}[1]{\textcolor[rgb]{0.77,0.63,0.00}{#1}}
\newcommand{\BaseNTok}[1]{\textcolor[rgb]{0.00,0.00,0.81}{#1}}
\newcommand{\BuiltInTok}[1]{#1}
\newcommand{\CharTok}[1]{\textcolor[rgb]{0.31,0.60,0.02}{#1}}
\newcommand{\CommentTok}[1]{\textcolor[rgb]{0.56,0.35,0.01}{\textit{#1}}}
\newcommand{\CommentVarTok}[1]{\textcolor[rgb]{0.56,0.35,0.01}{\textbf{\textit{#1}}}}
\newcommand{\ConstantTok}[1]{\textcolor[rgb]{0.00,0.00,0.00}{#1}}
\newcommand{\ControlFlowTok}[1]{\textcolor[rgb]{0.13,0.29,0.53}{\textbf{#1}}}
\newcommand{\DataTypeTok}[1]{\textcolor[rgb]{0.13,0.29,0.53}{#1}}
\newcommand{\DecValTok}[1]{\textcolor[rgb]{0.00,0.00,0.81}{#1}}
\newcommand{\DocumentationTok}[1]{\textcolor[rgb]{0.56,0.35,0.01}{\textbf{\textit{#1}}}}
\newcommand{\ErrorTok}[1]{\textcolor[rgb]{0.64,0.00,0.00}{\textbf{#1}}}
\newcommand{\ExtensionTok}[1]{#1}
\newcommand{\FloatTok}[1]{\textcolor[rgb]{0.00,0.00,0.81}{#1}}
\newcommand{\FunctionTok}[1]{\textcolor[rgb]{0.00,0.00,0.00}{#1}}
\newcommand{\ImportTok}[1]{#1}
\newcommand{\InformationTok}[1]{\textcolor[rgb]{0.56,0.35,0.01}{\textbf{\textit{#1}}}}
\newcommand{\KeywordTok}[1]{\textcolor[rgb]{0.13,0.29,0.53}{\textbf{#1}}}
\newcommand{\NormalTok}[1]{#1}
\newcommand{\OperatorTok}[1]{\textcolor[rgb]{0.81,0.36,0.00}{\textbf{#1}}}
\newcommand{\OtherTok}[1]{\textcolor[rgb]{0.56,0.35,0.01}{#1}}
\newcommand{\PreprocessorTok}[1]{\textcolor[rgb]{0.56,0.35,0.01}{\textit{#1}}}
\newcommand{\RegionMarkerTok}[1]{#1}
\newcommand{\SpecialCharTok}[1]{\textcolor[rgb]{0.00,0.00,0.00}{#1}}
\newcommand{\SpecialStringTok}[1]{\textcolor[rgb]{0.31,0.60,0.02}{#1}}
\newcommand{\StringTok}[1]{\textcolor[rgb]{0.31,0.60,0.02}{#1}}
\newcommand{\VariableTok}[1]{\textcolor[rgb]{0.00,0.00,0.00}{#1}}
\newcommand{\VerbatimStringTok}[1]{\textcolor[rgb]{0.31,0.60,0.02}{#1}}
\newcommand{\WarningTok}[1]{\textcolor[rgb]{0.56,0.35,0.01}{\textbf{\textit{#1}}}}
\usepackage{graphicx}
\makeatletter
\def\maxwidth{\ifdim\Gin@nat@width>\linewidth\linewidth\else\Gin@nat@width\fi}
\def\maxheight{\ifdim\Gin@nat@height>\textheight\textheight\else\Gin@nat@height\fi}
\makeatother
% Scale images if necessary, so that they will not overflow the page
% margins by default, and it is still possible to overwrite the defaults
% using explicit options in \includegraphics[width, height, ...]{}
\setkeys{Gin}{width=\maxwidth,height=\maxheight,keepaspectratio}
% Set default figure placement to htbp
\makeatletter
\def\fps@figure{htbp}
\makeatother
\setlength{\emergencystretch}{3em} % prevent overfull lines
\providecommand{\tightlist}{%
  \setlength{\itemsep}{0pt}\setlength{\parskip}{0pt}}
\setcounter{secnumdepth}{-\maxdimen} % remove section numbering
\ifluatex
  \usepackage{selnolig}  % disable illegal ligatures
\fi

\title{Assignment5\_HC}
\author{Kunal Sharma}
\date{April 25, 2021}

\begin{document}
\maketitle

\begin{Shaded}
\begin{Highlighting}[]
\FunctionTok{library}\NormalTok{(cluster)}
\FunctionTok{library}\NormalTok{(caret)}
\end{Highlighting}
\end{Shaded}

\begin{verbatim}
## Loading required package: lattice
\end{verbatim}

\begin{verbatim}
## Loading required package: ggplot2
\end{verbatim}

\begin{Shaded}
\begin{Highlighting}[]
\FunctionTok{library}\NormalTok{(dendextend)}
\end{Highlighting}
\end{Shaded}

\begin{verbatim}
## 
## ---------------------
## Welcome to dendextend version 1.14.0
## Type citation('dendextend') for how to cite the package.
## 
## Type browseVignettes(package = 'dendextend') for the package vignette.
## The github page is: https://github.com/talgalili/dendextend/
## 
## Suggestions and bug-reports can be submitted at: https://github.com/talgalili/dendextend/issues
## Or contact: <tal.galili@gmail.com>
## 
##  To suppress this message use:  suppressPackageStartupMessages(library(dendextend))
## ---------------------
\end{verbatim}

\begin{verbatim}
## 
## Attaching package: 'dendextend'
\end{verbatim}

\begin{verbatim}
## The following object is masked from 'package:stats':
## 
##     cutree
\end{verbatim}

\begin{Shaded}
\begin{Highlighting}[]
\FunctionTok{library}\NormalTok{(factoextra)}
\end{Highlighting}
\end{Shaded}

\begin{verbatim}
## Welcome! Want to learn more? See two factoextra-related books at https://goo.gl/ve3WBa
\end{verbatim}

\begin{Shaded}
\begin{Highlighting}[]
\FunctionTok{library}\NormalTok{(purrr)}
\end{Highlighting}
\end{Shaded}

\begin{verbatim}
## 
## Attaching package: 'purrr'
\end{verbatim}

\begin{verbatim}
## The following object is masked from 'package:caret':
## 
##     lift
\end{verbatim}

Apply hierarchical clustering to the data using Euclidean distance to
the normalized measurements. Use Agnes to compare the clustering from
single linkage, complete linkage, average linkage, and Ward. Choose the
best method.

\begin{Shaded}
\begin{Highlighting}[]
\NormalTok{c\_data }\OtherTok{\textless{}{-}} \FunctionTok{read.csv}\NormalTok{(}\StringTok{"A:/DATA\_SETS/Cereals.csv"}\NormalTok{)}
\FunctionTok{sum}\NormalTok{(}\FunctionTok{is.na}\NormalTok{(c\_data))}
\end{Highlighting}
\end{Shaded}

\begin{verbatim}
## [1] 4
\end{verbatim}

\begin{Shaded}
\begin{Highlighting}[]
\NormalTok{c\_data }\OtherTok{\textless{}{-}} \FunctionTok{na.omit}\NormalTok{(c\_data)}
\NormalTok{c\_data }\OtherTok{\textless{}{-}}\NormalTok{ c\_data[,}\DecValTok{4}\SpecialCharTok{:}\DecValTok{16}\NormalTok{]}
\NormalTok{c\_data }\OtherTok{\textless{}{-}} \FunctionTok{scale}\NormalTok{(c\_data, }\AttributeTok{center =}\NormalTok{ T, }\AttributeTok{scale =}\NormalTok{ T)}

\FunctionTok{set.seed}\NormalTok{(}\DecValTok{123}\NormalTok{)}

\CommentTok{\# Dissimilarity matrix}
\NormalTok{euclidean\_dist }\OtherTok{\textless{}{-}} \FunctionTok{dist}\NormalTok{(c\_data, }\AttributeTok{method =} \StringTok{"euclidean"}\NormalTok{)}


\NormalTok{method }\OtherTok{\textless{}{-}} \FunctionTok{c}\NormalTok{( }\StringTok{"average"}\NormalTok{, }\StringTok{"single"}\NormalTok{, }\StringTok{"complete"}\NormalTok{, }\StringTok{"ward"}\NormalTok{)}
\FunctionTok{names}\NormalTok{(method) }\OtherTok{\textless{}{-}} \FunctionTok{c}\NormalTok{( }\StringTok{"average"}\NormalTok{, }\StringTok{"single"}\NormalTok{, }\StringTok{"complete"}\NormalTok{, }\StringTok{"ward"}\NormalTok{)}
\NormalTok{ac\_values }\OtherTok{\textless{}{-}} \ControlFlowTok{function}\NormalTok{(x) \{}
  
  \FunctionTok{agnes}\NormalTok{(euclidean\_dist, }\AttributeTok{method =}\NormalTok{ x)}\SpecialCharTok{$}\NormalTok{ac}
\NormalTok{\}}

\FunctionTok{map\_dbl}\NormalTok{(method, ac\_values)}
\end{Highlighting}
\end{Shaded}

\begin{verbatim}
##   average    single  complete      ward 
## 0.7766075 0.6067859 0.8353712 0.9046042
\end{verbatim}

\begin{Shaded}
\begin{Highlighting}[]
\CommentTok{\#The agglomerative coefficient obtained by Ward\textquotesingle{}s method is the largest. }
\CommentTok{\#Let\textquotesingle{}s take a peek at the dendogram.}

\NormalTok{hc\_ward }\OtherTok{\textless{}{-}} \FunctionTok{agnes}\NormalTok{(euclidean\_dist, }\AttributeTok{method =} \StringTok{"ward"}\NormalTok{)}
\FunctionTok{pltree}\NormalTok{(hc\_ward, }\AttributeTok{cex =} \FloatTok{0.5}\NormalTok{, }\AttributeTok{hang =} \SpecialCharTok{{-}}\DecValTok{1}\NormalTok{, }\AttributeTok{main =} \StringTok{"Dendrogram of agnes for ward"}\NormalTok{)}
\end{Highlighting}
\end{Shaded}

\includegraphics{Assignment5_HC__files/figure-latex/unnamed-chunk-2-1.pdf}
How many clusters would you choose?

\begin{Shaded}
\begin{Highlighting}[]
\CommentTok{\#install.packages("NbClust")}

\NormalTok{hc\_ward }\OtherTok{\textless{}{-}} \FunctionTok{agnes}\NormalTok{(euclidean\_dist, }\AttributeTok{method =} \StringTok{"ward"}\NormalTok{)}
\FunctionTok{pltree}\NormalTok{(hc\_ward, }\AttributeTok{cex =} \FloatTok{0.5}\NormalTok{, }\AttributeTok{hang =} \SpecialCharTok{{-}}\DecValTok{1}\NormalTok{, }\AttributeTok{main =} \StringTok{"Dendrogram of agnes for ward"}\NormalTok{)}

\CommentTok{\#install.packages("NbClust")}
\FunctionTok{library}\NormalTok{(NbClust)}
\NormalTok{num\_of\_clust }\OtherTok{=} \FunctionTok{NbClust}\NormalTok{(c\_data, }\AttributeTok{distance =} \StringTok{"euclidean"}\NormalTok{, }\AttributeTok{min.nc =} \DecValTok{5}\NormalTok{, }\AttributeTok{max.nc =} \DecValTok{10}\NormalTok{, }\AttributeTok{method =} \StringTok{"ward.D"}\NormalTok{,}\AttributeTok{index =} \StringTok{\textquotesingle{}dunn\textquotesingle{}}\NormalTok{)}
\NormalTok{num\_of\_clust}\SpecialCharTok{$}\NormalTok{Best.nc}
\end{Highlighting}
\end{Shaded}

\begin{verbatim}
## Number_clusters     Value_Index 
##          7.0000          0.2604
\end{verbatim}

\begin{Shaded}
\begin{Highlighting}[]
\CommentTok{\#After checking NbClust value for best number of clusters, the best fits is with K=7}
\FunctionTok{rect.hclust}\NormalTok{(hc\_ward, }\AttributeTok{k =} \DecValTok{7}\NormalTok{, }\AttributeTok{border =} \DecValTok{2}\SpecialCharTok{:}\DecValTok{10}\NormalTok{)}
\end{Highlighting}
\end{Shaded}

\includegraphics{Assignment5_HC__files/figure-latex/unnamed-chunk-3-1.pdf}

\begin{Shaded}
\begin{Highlighting}[]
\NormalTok{clust\_comp }\OtherTok{\textless{}{-}} \FunctionTok{cutree}\NormalTok{(hc\_ward, }\AttributeTok{k =} \DecValTok{7}\NormalTok{)}
\NormalTok{temp3 }\OtherTok{\textless{}{-}} \FunctionTok{cbind}\NormalTok{(}\FunctionTok{as.data.frame}\NormalTok{(}\FunctionTok{cbind}\NormalTok{(c\_data,clust\_comp)))}
\end{Highlighting}
\end{Shaded}

Comment on the structure of the clusters and on their stability. Hint:
To check stability, partition the data and see how well clusters formed
based on one part apply to the other part. To do this: ● Cluster
partition A ● Use the cluster centroids from A to assign each record in
partition B (each record is assigned to the cluster with the closest
centroid). ● Assess how consistent the cluster assignments are compared
to the assignments based on all the data

\begin{Shaded}
\begin{Highlighting}[]
\NormalTok{c\_data }\OtherTok{\textless{}{-}} \FunctionTok{read.csv}\NormalTok{(}\StringTok{"A:/DATA\_SETS/Cereals.csv"}\NormalTok{)}
\FunctionTok{sum}\NormalTok{(}\FunctionTok{is.na}\NormalTok{(c\_data))}
\end{Highlighting}
\end{Shaded}

\begin{verbatim}
## [1] 4
\end{verbatim}

\begin{Shaded}
\begin{Highlighting}[]
\NormalTok{c\_data }\OtherTok{\textless{}{-}} \FunctionTok{na.omit}\NormalTok{(c\_data)}
\NormalTok{c\_data }\OtherTok{\textless{}{-}}\NormalTok{ c\_data[,}\DecValTok{4}\SpecialCharTok{:}\DecValTok{16}\NormalTok{]}
\CommentTok{\# Creating Partitions for into two data  }
\NormalTok{c\_partition\_A }\OtherTok{\textless{}{-}}\NormalTok{ c\_data[}\DecValTok{1}\SpecialCharTok{:}\DecValTok{37}\NormalTok{,]}
\NormalTok{c\_partition\_B }\OtherTok{\textless{}{-}}\NormalTok{ c\_data[}\DecValTok{38}\SpecialCharTok{:}\DecValTok{74}\NormalTok{,]}
\NormalTok{c\_partition\_A }\OtherTok{\textless{}{-}} \FunctionTok{scale}\NormalTok{(c\_partition\_A, }\AttributeTok{center =}\NormalTok{ T, }\AttributeTok{scale =}\NormalTok{ T)}
\NormalTok{c\_partition\_B }\OtherTok{\textless{}{-}} \FunctionTok{scale}\NormalTok{(c\_partition\_B, }\AttributeTok{center =}\NormalTok{ T, }\AttributeTok{scale =}\NormalTok{ T)}

\NormalTok{euclidean\_dist\_partition\_A }\OtherTok{\textless{}{-}} \FunctionTok{dist}\NormalTok{(c\_partition\_A, }\AttributeTok{method =} \StringTok{"euclidean"}\NormalTok{)}

\FunctionTok{names}\NormalTok{(method) }\OtherTok{\textless{}{-}} \FunctionTok{c}\NormalTok{( }\StringTok{"average"}\NormalTok{, }\StringTok{"single"}\NormalTok{, }\StringTok{"complete"}\NormalTok{, }\StringTok{"ward"}\NormalTok{)}
\NormalTok{ac\_values1 }\OtherTok{\textless{}{-}} \ControlFlowTok{function}\NormalTok{(x) \{}
  \FunctionTok{agnes}\NormalTok{(euclidean\_dist\_partition\_A, }\AttributeTok{method =}\NormalTok{ x)}\SpecialCharTok{$}\NormalTok{ac}
\NormalTok{\}}
\FunctionTok{map\_dbl}\NormalTok{(method, ac\_values1)}
\end{Highlighting}
\end{Shaded}

\begin{verbatim}
##   average    single  complete      ward 
## 0.7091020 0.6724675 0.7706708 0.8570846
\end{verbatim}

\begin{Shaded}
\begin{Highlighting}[]
\CommentTok{\#The agglomerative coefficient obtained by Ward\textquotesingle{}s method is the largest. }
\CommentTok{\#Let\textquotesingle{}s take a peek at the dendogram.}
\FunctionTok{set.seed}\NormalTok{(}\DecValTok{123}\NormalTok{)}
\NormalTok{hc\_ward\_partition\_A }\OtherTok{\textless{}{-}} \FunctionTok{agnes}\NormalTok{(euclidean\_dist\_partition\_A, }\AttributeTok{method =} \StringTok{"ward"}\NormalTok{)}
\FunctionTok{pltree}\NormalTok{(hc\_ward\_partition\_A, }\AttributeTok{cex =} \FloatTok{0.5}\NormalTok{, }\AttributeTok{hang =} \SpecialCharTok{{-}}\DecValTok{1}\NormalTok{, }\AttributeTok{main =} \StringTok{"Dendrogram of agnes for ward"}\NormalTok{)}
\end{Highlighting}
\end{Shaded}

\includegraphics{Assignment5_HC__files/figure-latex/unnamed-chunk-5-1.pdf}

\begin{Shaded}
\begin{Highlighting}[]
\NormalTok{clust\_comp\_partition\_A }\OtherTok{\textless{}{-}} \FunctionTok{cutree}\NormalTok{(hc\_ward\_partition\_A, }\AttributeTok{k =} \DecValTok{7}\NormalTok{)}

\NormalTok{result}\OtherTok{\textless{}{-}}\FunctionTok{as.data.frame}\NormalTok{(}\FunctionTok{cbind}\NormalTok{(c\_partition\_A,clust\_comp\_partition\_A))}

\CommentTok{\#result[result$clust\_comp\_partition\_A==1,]}
\CommentTok{\#center1\textless{}{-}colMeans(result[result$clust\_comp\_partition\_A==1,])}
\NormalTok{klust }\OtherTok{\textless{}{-}} \DecValTok{1}\SpecialCharTok{:}\DecValTok{7}
\ControlFlowTok{for}\NormalTok{ (i }\ControlFlowTok{in}\NormalTok{ klust) \{}
  \FunctionTok{assign}\NormalTok{(}\FunctionTok{paste0}\NormalTok{(}\StringTok{"center\_"}\NormalTok{,i), }\FunctionTok{colMeans}\NormalTok{(result[result}\SpecialCharTok{$}\NormalTok{clust\_comp\_partition\_A}\SpecialCharTok{==}\NormalTok{i,]))}
\NormalTok{\}}

\NormalTok{centroids }\OtherTok{\textless{}{-}} \FunctionTok{rbind}\NormalTok{(center\_1,center\_2,center\_3,center\_4,center\_5,center\_6,center\_7}
\NormalTok{                   )}

\NormalTok{combined }\OtherTok{\textless{}{-}} \FunctionTok{as.data.frame}\NormalTok{(}\FunctionTok{rbind}\NormalTok{(centroids[,}\SpecialCharTok{{-}}\DecValTok{14}\NormalTok{], c\_partition\_B))}
\NormalTok{temp1}\OtherTok{\textless{}{-}}\FunctionTok{get\_dist}\NormalTok{(combined)}
\NormalTok{temp2}\OtherTok{\textless{}{-}}\FunctionTok{as.matrix}\NormalTok{(temp1)}
\NormalTok{data1}\OtherTok{\textless{}{-}}\FunctionTok{data.frame}\NormalTok{(}\AttributeTok{data=}\FunctionTok{seq}\NormalTok{(}\DecValTok{1}\NormalTok{,}\FunctionTok{nrow}\NormalTok{(c\_partition\_B),}\DecValTok{1}\NormalTok{),}\AttributeTok{clusters=}\FunctionTok{rep}\NormalTok{(}\DecValTok{0}\NormalTok{,}\FunctionTok{nrow}\NormalTok{(c\_partition\_B)))}
\ControlFlowTok{for}\NormalTok{(i }\ControlFlowTok{in} \DecValTok{1}\SpecialCharTok{:}\FunctionTok{nrow}\NormalTok{(c\_partition\_B))}
\NormalTok{\{}
\NormalTok{  data1[i,}\DecValTok{2}\NormalTok{]}\OtherTok{\textless{}{-}}\FunctionTok{which.min}\NormalTok{(temp2[i}\SpecialCharTok{+}\DecValTok{7}\NormalTok{,}\DecValTok{1}\SpecialCharTok{:}\DecValTok{7}\NormalTok{])}
\NormalTok{\}}

\FunctionTok{cbind}\NormalTok{(temp3}\SpecialCharTok{$}\NormalTok{clust\_comp[}\DecValTok{38}\SpecialCharTok{:}\DecValTok{74}\NormalTok{],data1}\SpecialCharTok{$}\NormalTok{clusters)}
\end{Highlighting}
\end{Shaded}

\begin{verbatim}
##       [,1] [,2]
##  [1,]    4    4
##  [2,]    5    6
##  [3,]    2    2
##  [4,]    3    3
##  [5,]    6    5
##  [6,]    2    2
##  [7,]    2    2
##  [8,]    4    4
##  [9,]    3    3
## [10,]    3    3
## [11,]    4    4
## [12,]    5    5
## [13,]    4    2
## [14,]    4    4
## [15,]    7    5
## [16,]    6    5
## [17,]    6    5
## [18,]    2    5
## [19,]    4    4
## [20,]    2    2
## [21,]    6    5
## [22,]    5    6
## [23,]    5    6
## [24,]    6    5
## [25,]    6    5
## [26,]    6    5
## [27,]    3    3
## [28,]    5    6
## [29,]    6    5
## [30,]    7    7
## [31,]    4    4
## [32,]    7    5
## [33,]    5    5
## [34,]    3    3
## [35,]    5    5
## [36,]    5    6
## [37,]    3    3
\end{verbatim}

\begin{Shaded}
\begin{Highlighting}[]
\FunctionTok{table}\NormalTok{(temp3}\SpecialCharTok{$}\NormalTok{clust\_comp[}\DecValTok{38}\SpecialCharTok{:}\DecValTok{74}\NormalTok{]}\SpecialCharTok{==}\NormalTok{data1}\SpecialCharTok{$}\NormalTok{clusters)}
\end{Highlighting}
\end{Shaded}

\begin{verbatim}
## 
## FALSE  TRUE 
##    17    20
\end{verbatim}

\begin{Shaded}
\begin{Highlighting}[]
\CommentTok{\#We get 17 FALSE and 20 TRUE, indicating that the model is only partly stable.}
\end{Highlighting}
\end{Shaded}

The elementary public schools would like to choose a set of cereals to
include in their daily cafeterias. Every day a different cereal is
offered, but all cereals should support a healthy diet. For this goal,
you are requested to find a cluster of ``healthy cereals.'' Should the
data be normalized? If not, how should they be used in the cluster
analysis?

\begin{Shaded}
\begin{Highlighting}[]
\NormalTok{Nutri\_cereal }\OtherTok{\textless{}{-}} \FunctionTok{na.omit}\NormalTok{(}\FunctionTok{read.csv}\NormalTok{(}\StringTok{"A:/DATA\_SETS/Cereals.csv"}\NormalTok{))}
\NormalTok{Nutri\_cereal}\OtherTok{\textless{}{-}} \FunctionTok{cbind}\NormalTok{(Nutri\_cereal,clust\_comp)}


\ControlFlowTok{for}\NormalTok{ (i }\ControlFlowTok{in} \DecValTok{1}\SpecialCharTok{:}\DecValTok{7}\NormalTok{)\{}
  \FunctionTok{assign}\NormalTok{(}\FunctionTok{paste0}\NormalTok{(}\StringTok{"Cluster"}\NormalTok{,i), }\FunctionTok{mean}\NormalTok{(Nutri\_cereal[Nutri\_cereal}\SpecialCharTok{$}\NormalTok{clust\_comp}\SpecialCharTok{==}\NormalTok{i,}\StringTok{"rating"}\NormalTok{]))}

\NormalTok{\}}

\NormalTok{a}\OtherTok{\textless{}{-}}\FunctionTok{cbind}\NormalTok{(Cluster1,Cluster2,Cluster3,Cluster4,Cluster5,Cluster6,Cluster7)}
\FunctionTok{paste}\NormalTok{(}\StringTok{"clearly cluster 1 has maximum rating"}\NormalTok{, }\FunctionTok{max}\NormalTok{(a),}\StringTok{" hence we\textquotesingle{}ll choose it"}\NormalTok{)}
\end{Highlighting}
\end{Shaded}

\begin{verbatim}
## [1] "clearly cluster 1 has maximum rating 73.8444633333333  hence we'll choose it"
\end{verbatim}

We must normalize data since we are using the distance metic algorithm.
Since the features of data vary, we must scale it to similar features.

\end{document}
