% Options for packages loaded elsewhere
\PassOptionsToPackage{unicode}{hyperref}
\PassOptionsToPackage{hyphens}{url}
%
\documentclass[
]{article}
\usepackage{lmodern}
\usepackage{amssymb,amsmath}
\usepackage{ifxetex,ifluatex}
\ifnum 0\ifxetex 1\fi\ifluatex 1\fi=0 % if pdftex
  \usepackage[T1]{fontenc}
  \usepackage[utf8]{inputenc}
  \usepackage{textcomp} % provide euro and other symbols
\else % if luatex or xetex
  \usepackage{unicode-math}
  \defaultfontfeatures{Scale=MatchLowercase}
  \defaultfontfeatures[\rmfamily]{Ligatures=TeX,Scale=1}
\fi
% Use upquote if available, for straight quotes in verbatim environments
\IfFileExists{upquote.sty}{\usepackage{upquote}}{}
\IfFileExists{microtype.sty}{% use microtype if available
  \usepackage[]{microtype}
  \UseMicrotypeSet[protrusion]{basicmath} % disable protrusion for tt fonts
}{}
\makeatletter
\@ifundefined{KOMAClassName}{% if non-KOMA class
  \IfFileExists{parskip.sty}{%
    \usepackage{parskip}
  }{% else
    \setlength{\parindent}{0pt}
    \setlength{\parskip}{6pt plus 2pt minus 1pt}}
}{% if KOMA class
  \KOMAoptions{parskip=half}}
\makeatother
\usepackage{xcolor}
\IfFileExists{xurl.sty}{\usepackage{xurl}}{} % add URL line breaks if available
\IfFileExists{bookmark.sty}{\usepackage{bookmark}}{\usepackage{hyperref}}
\hypersetup{
  pdftitle={FinalProjectML},
  pdfauthor={Kunal Sharma},
  hidelinks,
  pdfcreator={LaTeX via pandoc}}
\urlstyle{same} % disable monospaced font for URLs
\usepackage[margin=1in]{geometry}
\usepackage{color}
\usepackage{fancyvrb}
\newcommand{\VerbBar}{|}
\newcommand{\VERB}{\Verb[commandchars=\\\{\}]}
\DefineVerbatimEnvironment{Highlighting}{Verbatim}{commandchars=\\\{\}}
% Add ',fontsize=\small' for more characters per line
\usepackage{framed}
\definecolor{shadecolor}{RGB}{248,248,248}
\newenvironment{Shaded}{\begin{snugshade}}{\end{snugshade}}
\newcommand{\AlertTok}[1]{\textcolor[rgb]{0.94,0.16,0.16}{#1}}
\newcommand{\AnnotationTok}[1]{\textcolor[rgb]{0.56,0.35,0.01}{\textbf{\textit{#1}}}}
\newcommand{\AttributeTok}[1]{\textcolor[rgb]{0.77,0.63,0.00}{#1}}
\newcommand{\BaseNTok}[1]{\textcolor[rgb]{0.00,0.00,0.81}{#1}}
\newcommand{\BuiltInTok}[1]{#1}
\newcommand{\CharTok}[1]{\textcolor[rgb]{0.31,0.60,0.02}{#1}}
\newcommand{\CommentTok}[1]{\textcolor[rgb]{0.56,0.35,0.01}{\textit{#1}}}
\newcommand{\CommentVarTok}[1]{\textcolor[rgb]{0.56,0.35,0.01}{\textbf{\textit{#1}}}}
\newcommand{\ConstantTok}[1]{\textcolor[rgb]{0.00,0.00,0.00}{#1}}
\newcommand{\ControlFlowTok}[1]{\textcolor[rgb]{0.13,0.29,0.53}{\textbf{#1}}}
\newcommand{\DataTypeTok}[1]{\textcolor[rgb]{0.13,0.29,0.53}{#1}}
\newcommand{\DecValTok}[1]{\textcolor[rgb]{0.00,0.00,0.81}{#1}}
\newcommand{\DocumentationTok}[1]{\textcolor[rgb]{0.56,0.35,0.01}{\textbf{\textit{#1}}}}
\newcommand{\ErrorTok}[1]{\textcolor[rgb]{0.64,0.00,0.00}{\textbf{#1}}}
\newcommand{\ExtensionTok}[1]{#1}
\newcommand{\FloatTok}[1]{\textcolor[rgb]{0.00,0.00,0.81}{#1}}
\newcommand{\FunctionTok}[1]{\textcolor[rgb]{0.00,0.00,0.00}{#1}}
\newcommand{\ImportTok}[1]{#1}
\newcommand{\InformationTok}[1]{\textcolor[rgb]{0.56,0.35,0.01}{\textbf{\textit{#1}}}}
\newcommand{\KeywordTok}[1]{\textcolor[rgb]{0.13,0.29,0.53}{\textbf{#1}}}
\newcommand{\NormalTok}[1]{#1}
\newcommand{\OperatorTok}[1]{\textcolor[rgb]{0.81,0.36,0.00}{\textbf{#1}}}
\newcommand{\OtherTok}[1]{\textcolor[rgb]{0.56,0.35,0.01}{#1}}
\newcommand{\PreprocessorTok}[1]{\textcolor[rgb]{0.56,0.35,0.01}{\textit{#1}}}
\newcommand{\RegionMarkerTok}[1]{#1}
\newcommand{\SpecialCharTok}[1]{\textcolor[rgb]{0.00,0.00,0.00}{#1}}
\newcommand{\SpecialStringTok}[1]{\textcolor[rgb]{0.31,0.60,0.02}{#1}}
\newcommand{\StringTok}[1]{\textcolor[rgb]{0.31,0.60,0.02}{#1}}
\newcommand{\VariableTok}[1]{\textcolor[rgb]{0.00,0.00,0.00}{#1}}
\newcommand{\VerbatimStringTok}[1]{\textcolor[rgb]{0.31,0.60,0.02}{#1}}
\newcommand{\WarningTok}[1]{\textcolor[rgb]{0.56,0.35,0.01}{\textbf{\textit{#1}}}}
\usepackage{graphicx,grffile}
\makeatletter
\def\maxwidth{\ifdim\Gin@nat@width>\linewidth\linewidth\else\Gin@nat@width\fi}
\def\maxheight{\ifdim\Gin@nat@height>\textheight\textheight\else\Gin@nat@height\fi}
\makeatother
% Scale images if necessary, so that they will not overflow the page
% margins by default, and it is still possible to overwrite the defaults
% using explicit options in \includegraphics[width, height, ...]{}
\setkeys{Gin}{width=\maxwidth,height=\maxheight,keepaspectratio}
% Set default figure placement to htbp
\makeatletter
\def\fps@figure{htbp}
\makeatother
\setlength{\emergencystretch}{3em} % prevent overfull lines
\providecommand{\tightlist}{%
  \setlength{\itemsep}{0pt}\setlength{\parskip}{0pt}}
\setcounter{secnumdepth}{-\maxdimen} % remove section numbering

\title{FinalProjectML}
\author{Kunal Sharma}
\date{07/05/2021}

\begin{document}
\maketitle

Problem Statement:

CRISA has traditionally segmented markets on the basis of purchaser
demographics. They would now like to segment the market based on two key
sets of variables more directly related to the purchase process and to
brand loyalty: 1. Purchase behavior (volume, frequency, susceptibility
to discounts, and brand loyalty) 2. Basis of purchase (price, selling
proposition) Doing so would allow CRISA to gain information about what
demographic attributes are associated with different purchase behaviors
and degrees of brand loyalty, and thus deploy promotion budgets more
effectively. More effective market segmentation would enable CRISA's
clients (in this case, a firm called IMRB) to design more cost-effective
promotions targeted at appropriate segments. Thus, multiple promotions
could be launched, each targeted at different market segments at
different times of the year. This would result in a more cost-effective
allocation of the promotion budget to different market segments. It
would also enable IMRB to design more effective customer reward systems
and thereby increase brand loyalty.

Question 1. Use k-means clustering to identify clusters of households
based on:

\begin{enumerate}
\def\labelenumi{\alph{enumi}.}
\item
  The variables that describe purchase behavior (including brand
  loyalty)
\item
  The variables that describe the basis for purchase
\item
  The variables that describe both purchase behavior and basis of
  purchase
\end{enumerate}

Note 1: How should k be chosen? Think about how the clusters would be
used. It is likely that the marketing efforts would support two to five
different promotional approaches. Note 2: How should the percentages of
total purchases comprised by various brands be treated? Isn't a customer
who buys all brand A just as loyal as a customer who buys all brand B?
What will be the effect on any distance measure of using the brand share
variables as is? Consider using a single derived variable.

\begin{enumerate}
\def\labelenumi{\arabic{enumi}.}
\setcounter{enumi}{1}
\item
  Select what you think is the best segmentation and comment on the
  characteristics (demographic, brand loyalty, and basis for purchase)
  of these clusters. (This information would be used to guide the
  development of advertising and promotional campaigns.)
\item
  Develop a model that classifies the data into these segments. Since
  this information would most likely be used in targeting direct-mail
  promotions, it would be useful to select a market segment that would
  be defined as a success in the classification model.
\end{enumerate}

\begin{Shaded}
\begin{Highlighting}[]
\KeywordTok{library}\NormalTok{(dplyr)}
\KeywordTok{library}\NormalTok{(ISLR)}
\KeywordTok{library}\NormalTok{(caret)}
\KeywordTok{library}\NormalTok{(factoextra)}
\KeywordTok{library}\NormalTok{(GGally)}
\KeywordTok{set.seed}\NormalTok{(}\DecValTok{123}\NormalTok{)}
\end{Highlighting}
\end{Shaded}

Reading And Cleaning the Data

\begin{Shaded}
\begin{Highlighting}[]
\NormalTok{BathSoap<-}\StringTok{ }\KeywordTok{read.csv}\NormalTok{(}\StringTok{"C://Users//admin//Downloads//BathSoap.csv"}\NormalTok{)}

\NormalTok{BSData <-}\StringTok{ }\KeywordTok{data.frame}\NormalTok{(}\KeywordTok{sapply}\NormalTok{(BathSoap, }\ControlFlowTok{function}\NormalTok{(x) }\KeywordTok{as.numeric}\NormalTok{(}\KeywordTok{gsub}\NormalTok{(}\StringTok{"%"}\NormalTok{, }\StringTok{""}\NormalTok{, x))))}
\end{Highlighting}
\end{Shaded}

For computing brand loyalty, we used data from branded purchases based
on the customer's purchase percentage on the Brand code, then found the
highest brand loyal percentage and compared it to the other 999 brand
purchases.

When a customer is loyal to a business, the Max Brand purchase
percentage is higher than the Other Brand purchase percentage. As a
consequence, the Customer's brand loyalty is created.

With k = 2, we use the kmeans clustering model to group the attributes
that define brand loyalty into ``Brand Loyal Customers'' and ``Not Brand
Loyal Customers.''

\begin{Shaded}
\begin{Highlighting}[]
\NormalTok{Loyal <-}\StringTok{ }\NormalTok{BSData[,}\DecValTok{23}\OperatorTok{:}\DecValTok{31}\NormalTok{]}

\NormalTok{Loyal}\OperatorTok{$}\NormalTok{MaxBrand <-}\StringTok{ }\KeywordTok{apply}\NormalTok{(Loyal,}\DecValTok{1}\NormalTok{,max)}

\NormalTok{BathSoapBrandLoyalty <-}\StringTok{ }\KeywordTok{cbind}\NormalTok{(BSData[,}\KeywordTok{c}\NormalTok{(}\DecValTok{19}\NormalTok{, }\DecValTok{13}\NormalTok{, }\DecValTok{15}\NormalTok{, }\DecValTok{12}\NormalTok{, }\DecValTok{31}\NormalTok{, }\DecValTok{14}\NormalTok{, }\DecValTok{16}\NormalTok{,}\DecValTok{20}\NormalTok{)], }\DataTypeTok{MaxLoyal =}\NormalTok{ Loyal}\OperatorTok{$}\NormalTok{MaxBrand)}

\NormalTok{BathSoapBrandLoyalty <-}\StringTok{ }\KeywordTok{scale}\NormalTok{(BathSoapBrandLoyalty)}

\NormalTok{K_model_}\DecValTok{2}\NormalTok{ <-}\StringTok{ }\KeywordTok{kmeans}\NormalTok{(BathSoapBrandLoyalty, }\DataTypeTok{centers =} \DecValTok{2}\NormalTok{, }\DataTypeTok{nstart =} \DecValTok{25}\NormalTok{)}

\NormalTok{BathSoapBrandLoyalty <-}\StringTok{ }\KeywordTok{cbind}\NormalTok{(BathSoapBrandLoyalty, }\DataTypeTok{Cluster =}\NormalTok{ K_model_}\DecValTok{2}\OperatorTok{$}\NormalTok{cluster)}

\KeywordTok{fviz_cluster}\NormalTok{(K_model_}\DecValTok{2}\NormalTok{, }\DataTypeTok{data =}\NormalTok{ BathSoapBrandLoyalty)}
\end{Highlighting}
\end{Shaded}

\includegraphics{FinalProjectML_files/figure-latex/unnamed-chunk-3-1.pdf}
Customers in Cluster 1 are Brand Loyal, while customers in Cluster 2 are
Brand Disloyal because they are unconcerned regarding products.

However, if we cluster them based on k = 4, we get the clusters shown
below.

\begin{Shaded}
\begin{Highlighting}[]
\NormalTok{K_model_}\DecValTok{2}\NormalTok{ <-}\StringTok{ }\KeywordTok{kmeans}\NormalTok{(BathSoapBrandLoyalty, }\DataTypeTok{centers =} \DecValTok{4}\NormalTok{, }\DataTypeTok{nstart =} \DecValTok{25}\NormalTok{)}

\NormalTok{BathSoapBrandLoyalty_}\DecValTok{4}\NormalTok{ <-}\StringTok{ }\KeywordTok{cbind}\NormalTok{(BathSoapBrandLoyalty[,}\OperatorTok{-}\DecValTok{10}\NormalTok{], }\DataTypeTok{Cluster =}\NormalTok{ K_model_}\DecValTok{2}\OperatorTok{$}\NormalTok{cluster)}

\KeywordTok{fviz_cluster}\NormalTok{(K_model_}\DecValTok{2}\NormalTok{, }\DataTypeTok{data =}\NormalTok{ BathSoapBrandLoyalty_}\DecValTok{4}\NormalTok{)}
\end{Highlighting}
\end{Shaded}

\includegraphics{FinalProjectML_files/figure-latex/unnamed-chunk-4-1.pdf}

Let's take a look at the data for consumer purchase conduct.

Here, for Selling Proposition we've considered all of the selling
propositions, selected the best of them, and compared them to show which
are the most successful selling propositions to consider for the Model.

\begin{Shaded}
\begin{Highlighting}[]
\NormalTok{BathSoap_SP <-}\StringTok{ }\NormalTok{BSData[,}\DecValTok{36}\OperatorTok{:}\DecValTok{46}\NormalTok{]}

\NormalTok{BathSoap_SP}\OperatorTok{$}\NormalTok{Max <-}\StringTok{ }\KeywordTok{apply}\NormalTok{(BathSoap_SP,}\DecValTok{1}\NormalTok{,max)}
\NormalTok{BathSoap_SP}\OperatorTok{$}\NormalTok{MaxBrand <-}\StringTok{ }\KeywordTok{colnames}\NormalTok{(BathSoap_SP)[}\KeywordTok{apply}\NormalTok{(BathSoap_SP,}\DecValTok{1}\NormalTok{,which.max)]}
\end{Highlighting}
\end{Shaded}

For the Price Catagories,catogories that are similar. Also the same can
be said for promotions.

\begin{Shaded}
\begin{Highlighting}[]
\NormalTok{PriceCategory <-}\StringTok{ }\NormalTok{BSData[,}\DecValTok{32}\OperatorTok{:}\DecValTok{35}\NormalTok{]}
\NormalTok{PriceCategory}\OperatorTok{$}\NormalTok{Max <-}\StringTok{ }\KeywordTok{apply}\NormalTok{(PriceCategory,}\DecValTok{1}\NormalTok{,max)}
\NormalTok{PriceCategory}\OperatorTok{$}\NormalTok{MaxBrand <-}\StringTok{ }\KeywordTok{colnames}\NormalTok{(PriceCategory)[}\KeywordTok{apply}\NormalTok{(PriceCategory,}\DecValTok{1}\NormalTok{,which.max)]}

\KeywordTok{table}\NormalTok{(PriceCategory}\OperatorTok{$}\NormalTok{MaxBrand)}
\end{Highlighting}
\end{Shaded}

\begin{verbatim}
## 
## Pr.Cat.1 Pr.Cat.2 Pr.Cat.3 Pr.Cat.4 
##      132      343       78       47
\end{verbatim}

\begin{Shaded}
\begin{Highlighting}[]
\NormalTok{Promo <-}\StringTok{ }\NormalTok{BSData[,}\DecValTok{20}\OperatorTok{:}\DecValTok{22}\NormalTok{]}
\NormalTok{Promo}\OperatorTok{$}\NormalTok{Max <-}\StringTok{ }\KeywordTok{apply}\NormalTok{(Promo,}\DecValTok{1}\NormalTok{,max)}
\NormalTok{Promo}\OperatorTok{$}\NormalTok{MaxBrand <-}\StringTok{ }\KeywordTok{colnames}\NormalTok{(Promo)[}\KeywordTok{apply}\NormalTok{(Promo,}\DecValTok{1}\NormalTok{,which.max)]}

\KeywordTok{table}\NormalTok{(Promo}\OperatorTok{$}\NormalTok{MaxBrand)}
\end{Highlighting}
\end{Shaded}

\begin{verbatim}
## 
##  Pur.Vol.No.Promo.... Pur.Vol.Other.Promo..     Pur.Vol.Promo.6.. 
##                   595                     1                     4
\end{verbatim}

As a result, we've only considered the more powerful Selling
Propositions when evaluating their effect. The same can be said for
promotions and price categories.

\begin{Shaded}
\begin{Highlighting}[]
\NormalTok{CustomerPurchaseBehaviour <-}\StringTok{ }\NormalTok{BSData[,}\KeywordTok{c}\NormalTok{(}\DecValTok{32}\NormalTok{,}\DecValTok{33}\NormalTok{,}\DecValTok{34}\NormalTok{,}\DecValTok{35}\NormalTok{,}\DecValTok{36}\NormalTok{,}\DecValTok{45}\NormalTok{)]}
\NormalTok{CustomerPurchaseBehaviour <-}\StringTok{ }\KeywordTok{scale}\NormalTok{(CustomerPurchaseBehaviour)}
\CommentTok{#View(CustomerPurchaseBehaviour)}

\KeywordTok{fviz_nbclust}\NormalTok{(CustomerPurchaseBehaviour, kmeans, }\DataTypeTok{method =} \StringTok{"silhouette"}\NormalTok{)}
\end{Highlighting}
\end{Shaded}

\includegraphics{FinalProjectML_files/figure-latex/unnamed-chunk-7-1.pdf}
The K means Clustering model is computed in order to measure the
customer's purchasing pattern. In this case, k = 4 will be used.

\begin{Shaded}
\begin{Highlighting}[]
\NormalTok{Purchase_K_model <-}\StringTok{ }\KeywordTok{kmeans}\NormalTok{(CustomerPurchaseBehaviour, }\DataTypeTok{centers =} \DecValTok{4}\NormalTok{, }\DataTypeTok{nstart =} \DecValTok{25}\NormalTok{)}

\NormalTok{CustomerPurchaseBehaviour <-}\StringTok{ }\KeywordTok{cbind}\NormalTok{(CustomerPurchaseBehaviour, }\DataTypeTok{Cluster =}\NormalTok{ Purchase_K_model}\OperatorTok{$}\NormalTok{cluster)}
\CommentTok{#View(CustomerPurchaseBehaviour)}

\KeywordTok{fviz_cluster}\NormalTok{(Purchase_K_model, }\DataTypeTok{data =}\NormalTok{ CustomerPurchaseBehaviour)}
\end{Highlighting}
\end{Shaded}

\includegraphics{FinalProjectML_files/figure-latex/unnamed-chunk-8-1.pdf}

Now we must understand the consumers' brand loyalty as well as their
purchasing behavior while developing a concept.

\begin{Shaded}
\begin{Highlighting}[]
\NormalTok{LoyalPurchase <-}\StringTok{ }\KeywordTok{cbind}\NormalTok{(BathSoapBrandLoyalty[,}\OperatorTok{-}\DecValTok{10}\NormalTok{], CustomerPurchaseBehaviour[,}\OperatorTok{-}\DecValTok{7}\NormalTok{])}

\KeywordTok{fviz_nbclust}\NormalTok{(LoyalPurchase, kmeans, }\DataTypeTok{method =} \StringTok{"silhouette"}\NormalTok{)}
\end{Highlighting}
\end{Shaded}

\includegraphics{FinalProjectML_files/figure-latex/unnamed-chunk-9-1.pdf}

\begin{Shaded}
\begin{Highlighting}[]
\NormalTok{K_Means_All <-}\StringTok{ }\KeywordTok{kmeans}\NormalTok{(LoyalPurchase, }\DataTypeTok{centers =} \DecValTok{4}\NormalTok{, }\DataTypeTok{nstart =} \DecValTok{25}\NormalTok{)}
\end{Highlighting}
\end{Shaded}

When plotting the model for k = 4 and k = 5, we can see that the aspects
can be resolved by simply using 4 clusters without drawing another 1. As
a result, we'll use k = 4 here.

\begin{Shaded}
\begin{Highlighting}[]
\NormalTok{LoyalPurchase <-}\StringTok{ }\KeywordTok{cbind}\NormalTok{(LoyalPurchase, }\DataTypeTok{Cluster =} \KeywordTok{as.data.frame}\NormalTok{(K_Means_All}\OperatorTok{$}\NormalTok{cluster))}
\NormalTok{clusters <-}\StringTok{ }\KeywordTok{matrix}\NormalTok{(}\KeywordTok{c}\NormalTok{(}\StringTok{"1"}\NormalTok{,}\StringTok{"2"}\NormalTok{,}\StringTok{"3"}\NormalTok{,}\StringTok{"4"}\NormalTok{),}\DataTypeTok{nrow =} \DecValTok{4}\NormalTok{)}
\NormalTok{LoyalPurchase_Centroids <-}\StringTok{ }\KeywordTok{cbind}\NormalTok{(clusters,}\KeywordTok{as.data.frame}\NormalTok{(K_Means_All}\OperatorTok{$}\NormalTok{centers))}

\KeywordTok{ggparcoord}\NormalTok{(LoyalPurchase_Centroids,}
           \DataTypeTok{columns =} \DecValTok{2}\OperatorTok{:}\DecValTok{16}\NormalTok{, }\DataTypeTok{groupColumn =} \DecValTok{1}\NormalTok{,}
           \DataTypeTok{showPoints =} \OtherTok{TRUE}\NormalTok{, }
           \DataTypeTok{title =} \StringTok{"Parallel Coordinate Plot for for Bathsoap Data - K = 4"}\NormalTok{,}
           \DataTypeTok{alphaLines =} \FloatTok{0.5}\NormalTok{)}
\end{Highlighting}
\end{Shaded}

\includegraphics{FinalProjectML_files/figure-latex/unnamed-chunk-10-1.pdf}

For each cluster, the Demographic result is computed.

We're simply attempting to decipher the demographic values of each
cluster.

\begin{Shaded}
\begin{Highlighting}[]
\NormalTok{Demographics <-}\KeywordTok{cbind}\NormalTok{(BSData[,}\DecValTok{2}\OperatorTok{:}\DecValTok{11}\NormalTok{], }\DataTypeTok{ClusterVal =}\NormalTok{ K_Means_All}\OperatorTok{$}\NormalTok{cluster)}

\NormalTok{Centre_}\DecValTok{1}\NormalTok{ <-}\StringTok{ }\KeywordTok{colMeans}\NormalTok{(Demographics[Demographics}\OperatorTok{$}\NormalTok{ClusterVal }\OperatorTok{==}\StringTok{ "1"}\NormalTok{,])}
\NormalTok{Centre_}\DecValTok{2}\NormalTok{ <-}\StringTok{ }\KeywordTok{colMeans}\NormalTok{(Demographics[Demographics}\OperatorTok{$}\NormalTok{ClusterVal }\OperatorTok{==}\StringTok{ "2"}\NormalTok{,])}
\NormalTok{Centre_}\DecValTok{3}\NormalTok{ <-}\StringTok{ }\KeywordTok{colMeans}\NormalTok{(Demographics[Demographics}\OperatorTok{$}\NormalTok{ClusterVal }\OperatorTok{==}\StringTok{ "3"}\NormalTok{,])}
\NormalTok{Centre_}\DecValTok{4}\NormalTok{ <-}\StringTok{ }\KeywordTok{colMeans}\NormalTok{(Demographics[Demographics}\OperatorTok{$}\NormalTok{ClusterVal }\OperatorTok{==}\StringTok{ "4"}\NormalTok{,])}

\NormalTok{Centroid <-}\StringTok{ }\KeywordTok{rbind}\NormalTok{(Centre_}\DecValTok{1}\NormalTok{, Centre_}\DecValTok{2}\NormalTok{, Centre_}\DecValTok{3}\NormalTok{, Centre_}\DecValTok{4}\NormalTok{)}

\KeywordTok{ggparcoord}\NormalTok{(Centroid,}
           \DataTypeTok{columns =} \KeywordTok{c}\NormalTok{(}\DecValTok{1}\NormalTok{,}\DecValTok{5}\NormalTok{,}\DecValTok{6}\NormalTok{,}\DecValTok{7}\NormalTok{,}\DecValTok{8}\NormalTok{), }\DataTypeTok{groupColumn =} \DecValTok{11}\NormalTok{,}
           \DataTypeTok{showPoints =} \OtherTok{TRUE}\NormalTok{, }
           \DataTypeTok{title =} \StringTok{"Demographic Metrics for Bathsoap Data Plotted in Parallel Coordinate Plot- K = 4"}\NormalTok{,}
           \DataTypeTok{alphaLines =} \FloatTok{0.5}\NormalTok{)}
\end{Highlighting}
\end{Shaded}

\includegraphics{FinalProjectML_files/figure-latex/unnamed-chunk-11-1.pdf}

We are presenting it in a barplot since there are a few attributes that
are categorical.

Plotting Eating Habit Frequency (Not Specified,Vegetarian Who Eats Eggs,
Vegetarian,Non-Vegetarian):

\begin{Shaded}
\begin{Highlighting}[]
\KeywordTok{barplot}\NormalTok{(}\KeywordTok{table}\NormalTok{(BSData}\OperatorTok{$}\NormalTok{FEH,K_Means_All}\OperatorTok{$}\NormalTok{cluster), }\DataTypeTok{xlab =} \StringTok{"Clusters"}\NormalTok{, }\DataTypeTok{ylab =} \StringTok{"Frequency of Eating Habit"}\NormalTok{, }\DataTypeTok{main =} \StringTok{"The Eating Habit Frequency for each cluster"}\NormalTok{,}\DataTypeTok{horiz=}\OtherTok{TRUE}\NormalTok{)}
\end{Highlighting}
\end{Shaded}

\includegraphics{FinalProjectML_files/figure-latex/unnamed-chunk-12-1.pdf}

Plotting the Frequency of Gender(NA, Male, Female):

\begin{Shaded}
\begin{Highlighting}[]
\KeywordTok{barplot}\NormalTok{(}\KeywordTok{table}\NormalTok{(BSData}\OperatorTok{$}\NormalTok{SEX,K_Means_All}\OperatorTok{$}\NormalTok{cluster), }\DataTypeTok{xlab =} \StringTok{"Clusters"}\NormalTok{, }\DataTypeTok{ylab =} \StringTok{"Frequency of Gender"}\NormalTok{, }\DataTypeTok{main =} \StringTok{"The Gender Frequency for each cluster"}\NormalTok{,}\DataTypeTok{horiz=}\OtherTok{TRUE}\NormalTok{)}
\end{Highlighting}
\end{Shaded}

\includegraphics{FinalProjectML_files/figure-latex/unnamed-chunk-13-1.pdf}
The female population has a higher purchasing rate, with females from
clusters 1 and 2 having the most females.

Plotting the Television Availability Frequency (Unspecified,
Availability, Not Available):

\begin{Shaded}
\begin{Highlighting}[]
\KeywordTok{barplot}\NormalTok{(}\KeywordTok{table}\NormalTok{(BSData}\OperatorTok{$}\NormalTok{CS, K_Means_All}\OperatorTok{$}\NormalTok{cluster), }\DataTypeTok{xlab =} \StringTok{"Clusters"}\NormalTok{, }\DataTypeTok{ylab =} \StringTok{"Frequency of Television availability"}\NormalTok{, }\DataTypeTok{main =} \StringTok{"The frequency of television availability"}\NormalTok{,}\DataTypeTok{horiz=}\OtherTok{TRUE}\NormalTok{)}
\end{Highlighting}
\end{Shaded}

\includegraphics{FinalProjectML_files/figure-latex/unnamed-chunk-14-1.pdf}

Since almost all viewers have access to television, having a promotional
offer on television can be effective in attracting customers.

Even, for those with codes 5 and 14, the Selling Proposition is strong.
These are strong perpositions, Henec, and they can be used in the
future.

Similarly, Price Catagory 1 and 2 have received positive feedback, so
they can be used again in the future to attract customers' attention.

Customers in cluster 1 often have a higher educational level, implying
that they have a decent job and that their email is reviewed often,
allowing for the mailing of promotions.

Now, if we consider the remediation for a higher profit on the soaps to
be sold, we can conclude that Cluster 3 customers are brand loyal. Thus,
any promotional deals for branded soaps can be sent to Cluster 3
customers.

Similarly, Cluster 4 consumers are ignorant about promotional deals, but
their sales are still high, so sending them a promotional email will not
help us make a lot of money. Rather, the Cluster 1 customers who buy
over the promotions are the most numerous, and they should be the mail
priority.

Where the Average Price is higher, the profit range can be expanded. As
a result, Cluster 3 consumers will concentrate on sending high-priced
products to the mail for recommendations.

\end{document}
