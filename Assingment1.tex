\documentclass[]{article}
\usepackage{lmodern}
\usepackage{amssymb,amsmath}
\usepackage{ifxetex,ifluatex}
\usepackage{fixltx2e} % provides \textsubscript
\ifnum 0\ifxetex 1\fi\ifluatex 1\fi=0 % if pdftex
  \usepackage[T1]{fontenc}
  \usepackage[utf8]{inputenc}
\else % if luatex or xelatex
  \ifxetex
    \usepackage{mathspec}
  \else
    \usepackage{fontspec}
  \fi
  \defaultfontfeatures{Ligatures=TeX,Scale=MatchLowercase}
\fi
% use upquote if available, for straight quotes in verbatim environments
\IfFileExists{upquote.sty}{\usepackage{upquote}}{}
% use microtype if available
\IfFileExists{microtype.sty}{%
\usepackage{microtype}
\UseMicrotypeSet[protrusion]{basicmath} % disable protrusion for tt fonts
}{}
\usepackage[margin=1in]{geometry}
\usepackage{hyperref}
\hypersetup{unicode=true,
            pdftitle={ASSINGMENT1},
            pdfborder={0 0 0},
            breaklinks=true}
\urlstyle{same}  % don't use monospace font for urls
\usepackage{color}
\usepackage{fancyvrb}
\newcommand{\VerbBar}{|}
\newcommand{\VERB}{\Verb[commandchars=\\\{\}]}
\DefineVerbatimEnvironment{Highlighting}{Verbatim}{commandchars=\\\{\}}
% Add ',fontsize=\small' for more characters per line
\usepackage{framed}
\definecolor{shadecolor}{RGB}{248,248,248}
\newenvironment{Shaded}{\begin{snugshade}}{\end{snugshade}}
\newcommand{\AlertTok}[1]{\textcolor[rgb]{0.94,0.16,0.16}{#1}}
\newcommand{\AnnotationTok}[1]{\textcolor[rgb]{0.56,0.35,0.01}{\textbf{\textit{#1}}}}
\newcommand{\AttributeTok}[1]{\textcolor[rgb]{0.77,0.63,0.00}{#1}}
\newcommand{\BaseNTok}[1]{\textcolor[rgb]{0.00,0.00,0.81}{#1}}
\newcommand{\BuiltInTok}[1]{#1}
\newcommand{\CharTok}[1]{\textcolor[rgb]{0.31,0.60,0.02}{#1}}
\newcommand{\CommentTok}[1]{\textcolor[rgb]{0.56,0.35,0.01}{\textit{#1}}}
\newcommand{\CommentVarTok}[1]{\textcolor[rgb]{0.56,0.35,0.01}{\textbf{\textit{#1}}}}
\newcommand{\ConstantTok}[1]{\textcolor[rgb]{0.00,0.00,0.00}{#1}}
\newcommand{\ControlFlowTok}[1]{\textcolor[rgb]{0.13,0.29,0.53}{\textbf{#1}}}
\newcommand{\DataTypeTok}[1]{\textcolor[rgb]{0.13,0.29,0.53}{#1}}
\newcommand{\DecValTok}[1]{\textcolor[rgb]{0.00,0.00,0.81}{#1}}
\newcommand{\DocumentationTok}[1]{\textcolor[rgb]{0.56,0.35,0.01}{\textbf{\textit{#1}}}}
\newcommand{\ErrorTok}[1]{\textcolor[rgb]{0.64,0.00,0.00}{\textbf{#1}}}
\newcommand{\ExtensionTok}[1]{#1}
\newcommand{\FloatTok}[1]{\textcolor[rgb]{0.00,0.00,0.81}{#1}}
\newcommand{\FunctionTok}[1]{\textcolor[rgb]{0.00,0.00,0.00}{#1}}
\newcommand{\ImportTok}[1]{#1}
\newcommand{\InformationTok}[1]{\textcolor[rgb]{0.56,0.35,0.01}{\textbf{\textit{#1}}}}
\newcommand{\KeywordTok}[1]{\textcolor[rgb]{0.13,0.29,0.53}{\textbf{#1}}}
\newcommand{\NormalTok}[1]{#1}
\newcommand{\OperatorTok}[1]{\textcolor[rgb]{0.81,0.36,0.00}{\textbf{#1}}}
\newcommand{\OtherTok}[1]{\textcolor[rgb]{0.56,0.35,0.01}{#1}}
\newcommand{\PreprocessorTok}[1]{\textcolor[rgb]{0.56,0.35,0.01}{\textit{#1}}}
\newcommand{\RegionMarkerTok}[1]{#1}
\newcommand{\SpecialCharTok}[1]{\textcolor[rgb]{0.00,0.00,0.00}{#1}}
\newcommand{\SpecialStringTok}[1]{\textcolor[rgb]{0.31,0.60,0.02}{#1}}
\newcommand{\StringTok}[1]{\textcolor[rgb]{0.31,0.60,0.02}{#1}}
\newcommand{\VariableTok}[1]{\textcolor[rgb]{0.00,0.00,0.00}{#1}}
\newcommand{\VerbatimStringTok}[1]{\textcolor[rgb]{0.31,0.60,0.02}{#1}}
\newcommand{\WarningTok}[1]{\textcolor[rgb]{0.56,0.35,0.01}{\textbf{\textit{#1}}}}
\usepackage{graphicx,grffile}
\makeatletter
\def\maxwidth{\ifdim\Gin@nat@width>\linewidth\linewidth\else\Gin@nat@width\fi}
\def\maxheight{\ifdim\Gin@nat@height>\textheight\textheight\else\Gin@nat@height\fi}
\makeatother
% Scale images if necessary, so that they will not overflow the page
% margins by default, and it is still possible to overwrite the defaults
% using explicit options in \includegraphics[width, height, ...]{}
\setkeys{Gin}{width=\maxwidth,height=\maxheight,keepaspectratio}
\IfFileExists{parskip.sty}{%
\usepackage{parskip}
}{% else
\setlength{\parindent}{0pt}
\setlength{\parskip}{6pt plus 2pt minus 1pt}
}
\setlength{\emergencystretch}{3em}  % prevent overfull lines
\providecommand{\tightlist}{%
  \setlength{\itemsep}{0pt}\setlength{\parskip}{0pt}}
\setcounter{secnumdepth}{0}
% Redefines (sub)paragraphs to behave more like sections
\ifx\paragraph\undefined\else
\let\oldparagraph\paragraph
\renewcommand{\paragraph}[1]{\oldparagraph{#1}\mbox{}}
\fi
\ifx\subparagraph\undefined\else
\let\oldsubparagraph\subparagraph
\renewcommand{\subparagraph}[1]{\oldsubparagraph{#1}\mbox{}}
\fi

%%% Use protect on footnotes to avoid problems with footnotes in titles
\let\rmarkdownfootnote\footnote%
\def\footnote{\protect\rmarkdownfootnote}

%%% Change title format to be more compact
\usepackage{titling}

% Create subtitle command for use in maketitle
\providecommand{\subtitle}[1]{
  \posttitle{
    \begin{center}\large#1\end{center}
    }
}

\setlength{\droptitle}{-2em}

  \title{ASSINGMENT1}
    \pretitle{\vspace{\droptitle}\centering\huge}
  \posttitle{\par}
    \author{}
    \preauthor{}\postauthor{}
    \date{}
    \predate{}\postdate{}
  

\begin{document}
\maketitle

\hypertarget{including-plots}{%
\subsection{Including Plots}\label{including-plots}}

You can also embed plots, for example:

\includegraphics{Assingment1_files/figure-latex/pressure-1.pdf}

\hypertarget{note-that-the-echo-false-parameter-was-added-to-the-code-chunk-to-prevent-printing-of-the-r-code-that-generated-the-plot.}{%
\subsection{\texorpdfstring{Note that the \texttt{echo\ =\ FALSE}
parameter was added to the code chunk to prevent printing of the R code
that generated the
plot.}{Note that the echo = FALSE parameter was added to the code chunk to prevent printing of the R code that generated the plot.}}\label{note-that-the-echo-false-parameter-was-added-to-the-code-chunk-to-prevent-printing-of-the-r-code-that-generated-the-plot.}}

output: word\_document: default html\_document: default ---

\hypertarget{kunal-sharma}{%
\subsection{KUNAL SHARMA}\label{kunal-sharma}}

\#Load dataset packages

\#import the excel file into the global environment

\begin{Shaded}
\begin{Highlighting}[]
\NormalTok{data }\OtherTok{\textless{}{-}} \FunctionTok{read.csv}\NormalTok{(}\StringTok{\textquotesingle{}A:/DATA\_SETS/a1{-}cereals.csv\textquotesingle{}}\NormalTok{)}
\end{Highlighting}
\end{Shaded}

\#head function gives the first 6 rows

\begin{Shaded}
\begin{Highlighting}[]
\FunctionTok{head}\NormalTok{(data)}
\end{Highlighting}
\end{Shaded}

\begin{verbatim}
##                    Cereal Manufacturer Type Calories Protein Fat Sodium
## 1 Apple Cinnamon Cheerios            G    C      110       2   2    180
## 2                 Basic 4            G    C      130       3   2    210
## 3                Cheerios            G    C      110       6   2    290
## 4   Cinnamon Toast Crunch            G    C      120       1   3    210
## 5                Clusters            G    C      110       3   2    140
## 6             Cocoa Puffs            G    C      110       1   1    180
##   Fiber Carbohydrates Sugars Shelf Potassium Vitamins Weight Cups
## 1   1.5          10.5     10     1        70       25   1.00 0.75
## 2   2.0          18.0      8     3       100       25   1.33 0.75
## 3   2.0          17.0      1     1       105       25   1.00 1.25
## 4   0.0          13.0      9     2        45       25   1.00 0.75
## 5   2.0          13.0      7     3       105       25   1.00 0.50
## 6   0.0          12.0     13     2        55       25   1.00 1.00
\end{verbatim}

\#tail function gives the last 6 rows

\begin{Shaded}
\begin{Highlighting}[]
\FunctionTok{tail}\NormalTok{(data)}
\end{Highlighting}
\end{Shaded}

\begin{verbatim}
##                               Cereal Manufacturer Type Calories Protein
## 72 Muesli Raisins, Peaches, & Pecans            R    C      150       4
## 73                         Rice Chex            R    C      110       1
## 74                        Wheat Chex            R    C      100       3
## 75                             Maypo            A    H      100       4
## 76            Cream of Wheat (Quick)            N    H      100       3
## 77                    Quaker Oatmeal            Q    H      100       5
##    Fat Sodium Fiber Carbohydrates Sugars Shelf Potassium Vitamins Weight
## 72   3    150   3.0            16     11     3       170       25     -1
## 73   0    240   0.0            23      2     1        30       25      1
## 74   1    230   3.0            17      3     1       115       25      1
## 75   1      0   0.0            16      3     2        95       25      1
## 76   0     80   1.0            21      0     2        -1        0      1
## 77   2      0   2.7            -1     -1     1       110        0      1
##     Cups
## 72 -1.00
## 73  1.13
## 74  0.67
## 75 -1.00
## 76  1.00
## 77  0.67
\end{verbatim}

\#Working with functions for Stats

\begin{Shaded}
\begin{Highlighting}[]
\FunctionTok{min}\NormalTok{(data}\SpecialCharTok{$}\NormalTok{Fat)}
\end{Highlighting}
\end{Shaded}

\begin{verbatim}
## [1] 0
\end{verbatim}

\begin{Shaded}
\begin{Highlighting}[]
\FunctionTok{max}\NormalTok{(data}\SpecialCharTok{$}\NormalTok{Fiber)}
\end{Highlighting}
\end{Shaded}

\begin{verbatim}
## [1] 14
\end{verbatim}

\begin{Shaded}
\begin{Highlighting}[]
\FunctionTok{range}\NormalTok{(data}\SpecialCharTok{$}\NormalTok{Vitamins)}
\end{Highlighting}
\end{Shaded}

\begin{verbatim}
## [1]   0 100
\end{verbatim}

\begin{Shaded}
\begin{Highlighting}[]
\FunctionTok{mean}\NormalTok{(data}\SpecialCharTok{$}\NormalTok{Weight)}
\end{Highlighting}
\end{Shaded}

\begin{verbatim}
## [1] 0.9776623
\end{verbatim}

\begin{Shaded}
\begin{Highlighting}[]
\FunctionTok{median}\NormalTok{(data}\SpecialCharTok{$}\NormalTok{Protein)}
\end{Highlighting}
\end{Shaded}

\begin{verbatim}
## [1] 3
\end{verbatim}

\#summary function will give the summary of the data

\begin{Shaded}
\begin{Highlighting}[]
\FunctionTok{summary}\NormalTok{(data)}
\end{Highlighting}
\end{Shaded}

\begin{verbatim}
##                        Cereal   Manufacturer Type      Calories    
##  100% Bran                : 1   A: 1         C:74   Min.   : 50.0  
##  100% Natural Bran        : 1   G:22         H: 3   1st Qu.:100.0  
##  All-Bran                 : 1   K:23                Median :110.0  
##  All-Bran with Extra Fiber: 1   N: 6                Mean   :106.9  
##  Almond Delight           : 1   P: 9                3rd Qu.:110.0  
##  Apple Cinnamon Cheerios  : 1   Q: 8                Max.   :160.0  
##  (Other)                  :71   R: 8                               
##     Protein           Fat            Sodium          Fiber       
##  Min.   :1.000   Min.   :0.000   Min.   :  0.0   Min.   : 0.000  
##  1st Qu.:2.000   1st Qu.:0.000   1st Qu.:130.0   1st Qu.: 1.000  
##  Median :3.000   Median :1.000   Median :180.0   Median : 2.000  
##  Mean   :2.545   Mean   :1.013   Mean   :159.7   Mean   : 2.152  
##  3rd Qu.:3.000   3rd Qu.:2.000   3rd Qu.:210.0   3rd Qu.: 3.000  
##  Max.   :6.000   Max.   :5.000   Max.   :320.0   Max.   :14.000  
##                                                                  
##  Carbohydrates      Sugars           Shelf         Potassium     
##  Min.   :-1.0   Min.   :-1.000   Min.   :1.000   Min.   : -1.00  
##  1st Qu.:12.0   1st Qu.: 3.000   1st Qu.:1.000   1st Qu.: 40.00  
##  Median :14.0   Median : 7.000   Median :2.000   Median : 90.00  
##  Mean   :14.6   Mean   : 6.922   Mean   :2.208   Mean   : 96.08  
##  3rd Qu.:17.0   3rd Qu.:11.000   3rd Qu.:3.000   3rd Qu.:120.00  
##  Max.   :23.0   Max.   :15.000   Max.   :3.000   Max.   :330.00  
##                                                                  
##     Vitamins          Weight             Cups        
##  Min.   :  0.00   Min.   :-1.0000   Min.   :-1.0000  
##  1st Qu.: 25.00   1st Qu.: 1.0000   1st Qu.: 0.5000  
##  Median : 25.00   Median : 1.0000   Median : 0.7500  
##  Mean   : 28.25   Mean   : 0.9777   Mean   : 0.5873  
##  3rd Qu.: 25.00   3rd Qu.: 1.0000   3rd Qu.: 1.0000  
##  Max.   :100.00   Max.   : 1.5000   Max.   : 1.5000  
## 
\end{verbatim}

\#str function will structure the data

\begin{Shaded}
\begin{Highlighting}[]
\FunctionTok{str}\NormalTok{(data)}
\end{Highlighting}
\end{Shaded}

\begin{verbatim}
## 'data.frame':    77 obs. of  15 variables:
##  $ Cereal       : Factor w/ 77 levels "100% Bran","100% Natural Bran",..: 6 8 12 13 14 15 19 23 32 38 ...
##  $ Manufacturer : Factor w/ 7 levels "A","G","K","N",..: 2 2 2 2 2 2 2 2 2 2 ...
##  $ Type         : Factor w/ 2 levels "C","H": 1 1 1 1 1 1 1 1 1 1 ...
##  $ Calories     : int  110 130 110 120 110 110 110 100 110 110 ...
##  $ Protein      : int  2 3 6 1 3 1 1 2 1 3 ...
##  $ Fat          : int  2 2 2 3 2 1 1 1 1 1 ...
##  $ Sodium       : int  180 210 290 210 140 180 180 140 280 250 ...
##  $ Fiber        : num  1.5 2 2 0 2 0 0 2 0 1.5 ...
##  $ Carbohydrates: num  10.5 18 17 13 13 12 12 11 15 11.5 ...
##  $ Sugars       : int  10 8 1 9 7 13 13 10 9 10 ...
##  $ Shelf        : int  1 3 1 2 3 2 2 3 2 1 ...
##  $ Potassium    : int  70 100 105 45 105 55 65 120 45 90 ...
##  $ Vitamins     : int  25 25 25 25 25 25 25 25 25 25 ...
##  $ Weight       : num  1 1.33 1 1 1 1 1 1 1 1 ...
##  $ Cups         : num  0.75 0.75 1.25 0.75 0.5 1 1 0.75 0.75 0.75 ...
\end{verbatim}

\begin{Shaded}
\begin{Highlighting}[]
\FunctionTok{plot}\NormalTok{(data)}
\end{Highlighting}
\end{Shaded}

\includegraphics{Assingment1_files/figure-latex/unnamed-chunk-8-1.pdf}
\#X-Y plot for two quantitative variables(Scatterplot)

\begin{Shaded}
\begin{Highlighting}[]
\FunctionTok{plot}\NormalTok{(}\AttributeTok{x=}\NormalTok{ data}\SpecialCharTok{$}\NormalTok{Carbohydrates, }\AttributeTok{y=}\NormalTok{data}\SpecialCharTok{$}\NormalTok{Sugars,}
     \AttributeTok{xlab =} \StringTok{"Carbs"}\NormalTok{,}
     \AttributeTok{ylab =} \StringTok{"SugarContent"}\NormalTok{,}
     \AttributeTok{xlim =} \FunctionTok{c}\NormalTok{(}\DecValTok{10}\NormalTok{,}\DecValTok{20}\NormalTok{),}
     \AttributeTok{ylim =} \FunctionTok{c}\NormalTok{(}\DecValTok{2}\NormalTok{,}\DecValTok{10}\NormalTok{),}
     \AttributeTok{main =} \StringTok{"Carbs with Sugars"}\NormalTok{)}
\end{Highlighting}
\end{Shaded}

\includegraphics{Assingment1_files/figure-latex/unnamed-chunk-9-1.pdf}
\#Add some Options

\begin{Shaded}
\begin{Highlighting}[]
\FunctionTok{plot}\NormalTok{(data}\SpecialCharTok{$}\NormalTok{Carbohydrates,data}\SpecialCharTok{$}\NormalTok{Sugars)}
\end{Highlighting}
\end{Shaded}

\includegraphics{Assingment1_files/figure-latex/unnamed-chunk-10-1.pdf}
\#Plotting a quantitative variable Audience\_Size

\begin{Shaded}
\begin{Highlighting}[]
\FunctionTok{plot}\NormalTok{(data}\SpecialCharTok{$}\NormalTok{Weight)}
\end{Highlighting}
\end{Shaded}

\includegraphics{Assingment1_files/figure-latex/unnamed-chunk-11-1.pdf}
\#LineplotHistogram,Boxplot

\begin{Shaded}
\begin{Highlighting}[]
\FunctionTok{hist}\NormalTok{(data}\SpecialCharTok{$}\NormalTok{Protein)}
\end{Highlighting}
\end{Shaded}

\includegraphics{Assingment1_files/figure-latex/unnamed-chunk-12-1.pdf}

\begin{Shaded}
\begin{Highlighting}[]
\FunctionTok{boxplot}\NormalTok{(data}\SpecialCharTok{$}\NormalTok{Potassium,data}\SpecialCharTok{$}\NormalTok{Vitamins)}
\end{Highlighting}
\end{Shaded}

\includegraphics{Assingment1_files/figure-latex/unnamed-chunk-12-2.pdf}

\begin{Shaded}
\begin{Highlighting}[]
\FunctionTok{plot}\NormalTok{(}\AttributeTok{x=}\NormalTok{ data}\SpecialCharTok{$}\NormalTok{Carbohydrates, }\AttributeTok{y=}\NormalTok{data}\SpecialCharTok{$}\NormalTok{Sugars,}

     \AttributeTok{xlab =} \StringTok{"Carbs"}\NormalTok{,}
     \AttributeTok{ylab =} \StringTok{"SugarContent"}\NormalTok{,}
     \AttributeTok{type=}\StringTok{"l"}\NormalTok{,}\AttributeTok{main=}\StringTok{"Example of Sugar and Carbs COntent"}\NormalTok{)}
\end{Highlighting}
\end{Shaded}

\includegraphics{Assingment1_files/figure-latex/unnamed-chunk-12-3.pdf}

\#The End


\end{document}
